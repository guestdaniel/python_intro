\documentclass{beamer}
\usepackage [english]{babel}
\usepackage{listings}
\lstset{language=Python,
	showstringspaces=false,
	tabsize=4,
	commentstyle=itshape,
	basicstyle=\ttfamily\tiny}

\title{Introduction to Python}
\subtitle{Installation, the basic idea, and some exploration}
\author{Daniel Guest}
\date{\today}

\AtBeginSection[]{
	\begin{frame}
	\vfill
	\centering
	\usebeamerfont{title}\insertsectionhead\par%
	\vfill
	\end{frame}
}

\begin{document}

\begin{frame}
	\titlepage
\end{frame}

\tableofcontents

\section{Installation}

\begin{frame}
\frametitle{Anaconda - the easy way}

\begin{enumerate}
	\item Check to see if you already have Python --- just type "Python" your terminal

	\item Download Anaconda for your operating system (use the Python 3.5 version)

	\begin{itemize}

		\item Windows: \url{https://repo.continuum.io/archive/Anaconda3-4.1.1-Windows-x86_64.exe} (64-bit) and \url{https://repo.continuum.io/archive/Anaconda3-4.1.1-Windows-x86.exe} (32-bit)
			
		\item Mac OS: \url{https://repo.continuum.io/archive/Anaconda3-4.1.1-MacOSX-x86_64.pkg}
	\end{itemize}
	\item Install Anaconda

	\item Anaconda is popular because it manages much of the difficult part of Python for you, and it includes most of the packages for scientific computing, data processing/visualization, and machine learning

	\item Install any packages not included with Anaconda through the Anaconda GUI
\end{enumerate}

\end{frame}

\begin{frame}
\frametitle{If you install the hard way...}

\begin{itemize}
	\item Install from \url{https://www.python.org/downloads/release/python-352/}

	\item Learn how to install packages from \url{https://packaging.python.org/installing/}

	\item You'll probably want...

		\begin{enumerate}
		\item NumPy --- MATLAB style matrix and signal processing

		\item Matplotlib --- plotting ala ggplot2

		\item pandas --- data frames similar to R's data frames

		\item SciPy --- all sorts of handy stuff for scientific computing

		\item PyMC3 --- Bayesian models in Python

		\item Theano --- Fast computation of complex mathematical expressions (used in deep learning)

		\item PyBrain --- "easy" neural networks

	\end{enumerate}
	\item Check out the "SciPy Stack" for a good group of useful packages.

\end{itemize}

\end{frame}

\section{Why Python?}

\begin{frame}
\frametitle{The design of Python}

\begin{itemize}
	\item Python is \emph{readable} --- the syntax is clean and designed for human eyes

	\item Python is \emph{interpreted} --- you can test small segments of code easily 

	\item Python is \emph{extensible} and \emph{modular} --- if basic Python doesn't do something you want, it's easy to install packages that do

	\item Python is \emph{easy} --- much of Python's scientific computing code was designed by scientists for scientists

	\item Python is \emph{fast} --- although not every built-in Python method is fast, there are many ways to speed up Python if you need it

	\item Python is \emph{expressive} --- your code will often be shorter and clearer
	
	\item Python is \emph{general purpose} --- you can do anything, all in one language
\end{itemize}
\end{frame}

\begin{frame}
\frametitle{The design of Python}
\begin{itemize}
	\item Together, all of these characteristics mean that Python will make your life easier!

	\item Many people will tell you that Python is \emph{natural} and \emph{intuitive}, or that Python is the easiest way to express their thoughts in code
	
	\item As I recently saw someone on a forum write, "Python makes programming fun again"

\end{itemize}
\end{frame}

\begin{frame}
\frametitle{The Python community}

\begin{itemize}
	\item One key advantage of Python is its community

	\item The community is \emph{large}, \emph{diverse}, and \emph{active}, so most areas of your interest are probably covered, and it's easy to get help

	\item Much of the community follows the PEP 8 style guide (\url{https://www.python.org/dev/peps/pep-0008/}), leading to consistency in how code is written across the community

	\item The vast majority of Python code is \emph{open source}, so you'll never have to pay for anything
\end{itemize}
\end{frame}

\section{Exploring Python Through the Terminal}

\begin{frame}
\frametitle{Starting Python}

\begin{itemize}
	\item If you have Python anywhere on your computer, you can simply run "python" in your terminal

	\item But, no one really uses the plain Python terminal! For this demo, we'll use IPython, an enhanced Python terminal

	\item To run IPython, just run "ipython" in your terminal or command prompt

	\item The IPython terminal works very similarly to the terminal in R or MATLAB
	
	\item We can talk about writing scripts and using integrated development environments (IDEs) in the future --- for now, we'll focus on getting the basics down!

	\item We'll move on once everyone has IPython running...
\end{itemize}
\end{frame}

\subsection{Variables}

\begin{frame}[fragile]
\frametitle{Variables}
\begin{lstlisting}
	In [1]: x = 5
	
	In [2]: x
	Out[2]: ?? 

	In [3]: x = x + 5

	In [4]: x 
	Out[4]: ??

	In [5]: x = "Hello"

	In [6]: x
	Out[6]: ??

	In [7]: y = " World!"

	In [8]: x + y
	Out[8]: ??

	In [9]: x = True

	In [10]: y = False

	In [11]: x and y
	Out[11]: ??

	In [12]: x or y
	Out[12]: ??

	In [13]: print(x)
	Out[13]: ??
\end{lstlisting}
\end{frame}

\begin{frame}[fragile]
\frametitle{Variables}
\begin{lstlisting}
	In [1]: x = 5
	
	In [2]: x
	Out[2]: 5 

	In [3]: x = x + 5

	In [4]: x 
	Out[4]: 10 

	In [5]: x = "Hello"

	In [6]: x
	Out[6]: "Hello" 

	In [7]: y = " World!"

	In [8]: x + y 
	Out[8]: "Hello world!" 

	In [9]: x = True

	In [10]: y = False

	In [11]: x and y
	Out[11]: False 

	In [12]: x or y
	Out[12]: True

	In [13]: print(x)
	Out[13]: True
\end{lstlisting}
\end{frame}

\begin{frame}[fragile]
\frametitle{Variables}
\begin{lstlisting}
	In [14]: x = 5

	In [15]: print(x)
	Out[15]: ??

	In [16]: y = "words"

	In [17]: print(x + y) 
	Out[17]: ??

	In [18]: print(str(x) + y)
	Out[18]: ??

	In [18]: print("I know at least " + str(x) + y)
	Out[18]: ??
\end{lstlisting}
\end{frame}

\begin{frame}[fragile]
\frametitle{Variables}
\begin{lstlisting}
	In [14]: x = 5

	In [15]: print(x)
	Out[15]: 5 

	In [16]: y = "words"

	In [17]: print(x + y) 
	Out[17]: ERROR 

	In [18]: print(str(x) + y)
	Out[18]: 5words 

	In [18]: print("I know at least " + str(x) + " " + y)
	Out[18]: "I know at least 5 words" 
\end{lstlisting}
\end{frame}

\begin{frame}
\frametitle{Variables: notes}
	\begin{enumerate}

		\item Values are assigned to a name by the = operator

		\item Variables are \emph{dynamic}, so they can store anything!

		\item While powerful, dynamic variables can lead to errors if you don't keep track of what they contain!

		\item String concatenation and boolean comparison is easy and readable

		\item Convert non-strings to strings with str()

		\item At least in terminal, just writing the name of a variable and passing it to the "print" function do the same thing

	\end{enumerate}
\end{frame}

\subsection{Strings}

\begin{frame}[fragile]
\frametitle{Strings}

\begin{lstlisting}
	In [19]: x = "Hello, how are you?"

	In [20]: x
	Out[20]: ??

	In [21]: x[0:5]
	Out[21]: ??

	In [22]: x[0:5] + " " + x[0:5] + "?"
	Out[22]: ??

	In [23]: len(x)
	Out[23]: ??

	In [24]: x.count("h")
	Out[24]: ??

	In [25]: x.find("h")
	Out[25]: ??

	In [26]: x[7]
	Out[26]: ?? 

	In [27]: x[x.find("a")]
	Out[27]: ?? 

	In [28]: x = "C:\Documents\lab\python\something_old.txt"

	In [29]: x = x[0:-4] + ".py"
	
	In [30]: x.replace("old", "new")
	Out[30]: ?? 

\end{lstlisting}
\end{frame}

\begin{frame}[fragile]
\frametitle{Strings}

\begin{lstlisting}
	In [19]: x = "Hello, how are you?"

	In [20]: x
	Out[20]: "Hello, how are you?" 

	In [21]: x[0:5]
	Out[21]: "Hello" 

	In [22]: x[0:5] + " " + x[0:5] + "?"
	Out[22]: "Hello Hello?" 

	In [23]: len(x)
	Out[23]: 19  

	In [24]: x.count("h")
	Out[24]: 1 

	In [25]: x.find("h")
	Out[25]: 7

	In [26]: x[7]
	Out[26]: "h"

	In [27]: x[x.find("a")]
	Out[27]: "a"

	In [28]: x = "C:\Documents\lab\python\something_old.txt"

	In [29]: x = x[0:-4] + ".py"
	
	In [30]: x.replace("old", "new")
	Out[30]: "C:\Documents\lab\python\something_new.py" 
\end{lstlisting}
\end{frame}

\begin{frame}
\frametitle{Strings}
	\begin{enumerate}
		\item Strings are very powerful and easy to manipulate in Python

		\item You can concatenate strings using the + operator

		\item You can index and slice strings in various ways --- we'll see more of this later

		\item There are many built-in functions to find and count characters and words inside of strings, or to replace parts of strings (we only scratched the surface here --- string manipulation is much easier in Python than R or MATLAB)

		\item You can check the length of a string with the len() --- this is used for most objects in Python
		\item Strings, and all other objects in Python, are indexed with [index] and not (index), which makes code more readable
	\end{enumerate}
\end{frame}

\subsection{Lists}

\begin{frame}[fragile]
\frametitle{Lists (or, where Python starts to get cool)}
\begin{lstlisting}
	In [28]: x = [1, 2, 3, 4, 5]

	In [29]: x
	Out[29]: ??

	In [30]: x[2]
	Out[30]: ??

	In [31]: x[0]
	Out[31]: ??

	In [32]: x[0:4]
	Out[32]: ??

	In [33]: x[-1]
	Out[33]: ??

	In [34]: x[-3:]
	Out[34]: ??

	In [35]: x[-4:-2]
	Out[35]: ??

	In [36]: x[5]
	Out[36]: ??

	In [37]: x.append(6)
	
	In [38]: x[5]
	Out[38]: ??
\end{lstlisting}
\end{frame}

\begin{frame}[fragile]
\frametitle{Lists (or, where Python starts to get cool)}
\begin{lstlisting}
	In [28]: x = [1, 2, 3, 4, 5]

	In [29]: x
	Out[29]: [1, 2, 3, 4, 5] 

	In [30]: x[2]
	Out[30]: 3

	In [31]: x[0]
	Out[31]: 1 

	In [32]: x[0:4]
	Out[32]: [1, 2, 3, 4] 

	In [33]: x[-1]
	Out[33]: 5 

	In [34]: x[-3:]
	Out[34]: [3, 4, 5] 

	In [35]: x[-4:-2]
	Out[35]: [2, 3]

	In [36]: x[5]
	Out[36]: ERROR 

	In [37]: x.append(6)
	
	In [38]: x[5]
	Out[38]: 6 
\end{lstlisting}
\end{frame}

\begin{frame}[fragile]
\frametitle{Lists}
\begin{lstlisting}
	In [43]: x = ["Joe"]
	
	In [44]: len(x)
	Out[44]: ??

	In [45]: x = ["Austin", "Houston", "Dallas", "San Antonio", "Fort Worth"]

	In [46]: len(x)
	Out[46]: ??

	In [47]: x.insert(1, "El Paso")

	In [48]: x[0:3]
	Out[48]: ?? 

	In [49]: x.index["San Antonio"]
	Out[49]: ??

	In [50]: x.sort()

	In [51]: x
	Out[51]: ??

	In [52]: x = ["9", 9, "Nine", ["9", 9, "Nine"], True]

	In [53]: x[0] + x[2]
	Out[53]: ??

	In [54]: x[3]
	Out[54]: ??

	In [55]: x[1] + x[3][1]
	Out[55]: ??
\end{lstlisting}
\end{frame}

\begin{frame}[fragile]
\frametitle{Lists}
\begin{lstlisting}
	In [43]: x = ["Joe"]
	
	In [44]: len(x)
	Out[44]: 1 

	In [45]: x = ["Austin", "Houston", "Dallas", "San Antonio", "Fort Worth"]

	In [46]: len(x)
	Out[46]: 5 

	In [47]: x.insert(1, "El Paso")

	In [48]: x[0:3]
	Out[48]: ["Austin", "El Paso", "Houston"] 

	In [49]: x.index["San Antonio"]
	Out[49]: 4 

	In [50]: x.sort()

	In [51]: x
	Out[51]: ["Austin", "Dallas", "El Paso", "Fort Worth", "Houston", "San Antonio"] 

	In [52]: x = ["9", 9, "Nine", ["9", 9, "Nine"], True]

	In [53]: x[0] + x[2]
	Out[53]: "9Nine" 

	In [54]: x[3]
	Out[54]: ["9", 9, "Nine"] 

	In [55]: x[1] + x[3][1]
	Out[55]: 18 
\end{lstlisting}
\end{frame}

\section{Practical Python}

\subsection{Modules}

\begin{frame}[fragile]
\frametitle{Importing modules}

Python, unlike MATLAB, has a small standard library. Although Python's standard library is very powerful, sometimes we need specific tools to get the job done. Python does this by allowing the user to import modules. We'll see how this is done below...

\begin{lstlisting}
	In [56]: x = [1, 4, 9, 16, 25]

	In [57]: sqrt(x[1])
	Out[57]: ?? 
	
	In [58]: import math

	In [59]: math
	Out[59]: ??

	In [60]: math.sqrt(x[1])
	Out[60]: ??

	In [61]: dir(math)
	Out[61]: ??

	In [62]: import math as m

	In [63]: m.sqrt(x[4])
	Out[63]: ??

	In [64]: from math import sqrt

	In [65]: sqrt(x[4])
	Out[65]: ??
\end{lstlisting}
\end{frame}

\begin{frame}
\frametitle{Modules --- notes}
	\begin{enumerate}
		\item Modules are bundles of related code

		\item You have to explicitly import modules to use them by using the \emph{import} keyword

		\item This might seem tedious at first, but it (a) allows for functions to share names, if they come from different modules, (b) makes it easy to import your own functions across different projects, and (c) makes larger projects more organized than in other languages 
	\end{enumerate}
\end{frame}

\subsection{A relatable example}

\begin{frame}
\frametitle{A relatable example --- the idea}
Suppose we want to...
	\begin{enumerate}
		\item Import a .wav file of a recorded vowel 

		\item Calculate the spectrum of the vowel 

		\item Plot the spectrum of the vowel and add labels
	\end{enumerate}
\end{frame}

\begin{frame}[fragile]
\frametitle{A relatable example --- the code}
\begin{lstlisting}
	import numpy as np
	import matplotlib.pyplot as plt
	from scipy.io import wavfile

	location = "/home/daniel/Documents/lab/vowgenface/vowels/unswapped/kabrii01.wav"

	[fs, vowel] = wavfile.read(location) 
	n = len(vowel)
	time_axis = np.arange(n)/fs
	vowel = vowel/np.max(np.abs(vowel))

	plt.plot(time_axis, vowel)
	plt.ylabel("Amplitude")
	plt.xlabel("Time")
	plt.show()

	log_spectrum = 20*np.log10(np.fft.rfft(vowel))
	dt = n/fs
	freq_axis = (np.arange(n)/dt)[0:len(log_spectrum)]

	plt.plot(freq_axis, log_spectrum)
	plt.ylabel("Amplitude")
	plt.xlabel("Frequency")
	plt.show()
\end{lstlisting}
\end{frame}

\begin{frame}
\frametitle{A relatable example --- notes}
	\begin{enumerate}
		\item Usually, modules are imported at the top of the code

		\item NumPy and PyPlot are designed to be familiar to MATLAB users (easy transition)
	\end{enumerate}
\end{frame}

\begin{frame}
\frametitle{An advanced example --- the idea}
\begin{enumerate}
	\item Monte Carlo Markov chain (MCMC) algorithms work by using a random walk to sample a posterior distrobution that cannot be solved analytically

	\item Suppose we want to see this random walk in action to develop a better understanding of what's really going on!

	\item We can simulate some data using NumPy or Pandas... 
	
	\item Then, we can use PyMC3 to create a Bayesian model...

	\item Finally, we can use matplotlib to make an animation of each step in action and save it to a video file for easy sharing!

	\item We're not going to write this example --- we're just going to look at the code for an idea of what Python can do very easily
\end{enumerate}
\end{frame}

\section{Thoughts and resources}

\begin{frame}
\frametitle{Some thoughts...}
	\begin{enumerate}
		\item I love Python, but even I don't use Python for everything --- I still prefer R+dplyr for data manipulation and R+ggplot2 for data visualization

		\item That said, Python is catching up in every area that it's behind, and is ahead in many areas

		\item Python is much better than MATLAB/R for machine learning and advanced Bayesian modeling, for example 

		\item Python enables you to focus on just one language, rather than being split between multiple languages

		\item Python is useful at every skill level and scale --- it's easy for a beginner, while still being powerful for an expert 
	\end{enumerate}
\end{frame}

\begin{frame}
\frametitle{Some resources...}
	\begin{enumerate}
		\item Type "help(function)" into the Python terminal to get built-in help!

		\item \url{https://www.python.org/about/gettingstarted/} --- great documentation!

		\item \url{https://wiki.python.org/moin/BeginnersGuide/Programmers} --- lots of tutorials, videos, guides, example code, etc.

		\item \url{https://docs.python.org/3/} --- formal documentation, for the technically adventurous
	\end{enumerate}
\end{frame}

\end{document}
